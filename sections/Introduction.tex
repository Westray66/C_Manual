\section{Introduction}

The C Programming Language first appeared in 1972. It was originally designed and developed by Dennis Ritchie. Since the first release the language has gone through a number of major revisions. Today's C language is much improved from the original but continues to retain many of the fundamental qualities and attributes that made it popular.\\ 

There are a number of modern competitors to the C language, such as \textit{Go} and \textit{Rust}. Despite these alternatives, C continues to have a strong following, especially among embedded and system-software engineers. There is a vast wealth of software available and it remains in the top most popular languages on GitHub. The popularity stems from three core attributes, namely simplicity, efficiency and flexibility.\\

Simplicity, the entire definition of the language fits on a few pages of Backus-Naur Form (BNF). If unfamiliar with BNF it is worth taking some time doing some research. BNF is a grammar used to describe programming languages. It has roots dating back to the release of \textit{Algol 58}.\\

Efficiency, the compilers are both mature and highly sophisticated. C converts extremely efficiently to the underlying hardware architecture. And likewise due to its popularity and importance many of today's hardware architectures have been directly influenced by C. A low-level compiled C program is significantly faster than high-level interpreted or abstracted languages such as \textit{Python} or \textit{JavaScript}.\\

Flexibility, the C language provides little in the way of runtime protection against errors. Compilers have improved in recent years but still allow dangerous activity. The dangerous side is important when interfacing with exotic hardware. It is this feature which allows C to be extremely powerful. Stability is in the hands of the programmer i.e. with great power comes great responsibility.\\

To emphasize these advantages, a paper released 2017 at the Software Language Engineering Conference called \textit{Energy Efficiency across Programming Languages} really highlighted the importance of the C programming language. C dominated the different categories and appeared in the top three languages for \textit{"Time \& Memory"}, \textit{"Energy \& Time"}, \textit{"Energy \& Memory"} and finally \textit{"Energy \& Time \& Memory"} efficiency. Other languages also in the top three were \textit{Go} and \textit{Pascal}. By comparison, \textit{Python} appears considerably lower in the efficiency-lists.\\

In this book we will concentrate on the C Programming Language (for Unix) and will highlight the parts of the language that are important to embedded and real-time engineers. Just to be clear this is about the C Language and not about \textit{C++}. \textit{C++} is a separate language with a different paradigm.\\ 

Lastly, and probably the most important point, C is great fun so go enjoy programming.
