\section{Miscellaneous}

This section contains some useful functions that don't fit into the framework but are really useful functions when programming. 

\subsection{system(...)}

\index{system(...)}

The \textit{system(...)} function is found in \textit{stdio.h}. It allows for interaction with the command line shell. This is useful when having to execute certain commands while under program control. Most of the commands available in Linux also have a program interface but sometimes it is easier to call the command directly via the host shell.\\

\begin{lstlisting}[language=C,showstringspaces=false, caption={File: system.c},captionpos=b,label=system]

 1 #include <stdio.h>
 2 #include <stdlib.h>
 3 
 4 int main(void)
 5 {
 6 FILE *HANDLE;
 7 char user[20];
 8 
 9   if (!system("whoami > me.dat"))
10     printf("success\n");
11   else
12     printf("failed\n");
13 
14 HANDLE=fopen("me.dat","r");
15 
16   if (HANDLE==NULL)
17   {
18   perror("file me.dat");
19   exit(EXIT_FAILURE);
20   }
21 
22 fscanf(HANDLE,"%s",user);
23 fclose(HANDLE);
24 
25 printf("user: %s\n",user);
26 
27 return 0;
28 }

INTERACTION

$ cc system.c
$ ./a.out
success
user: proteus
$

\end{lstlisting}

Listing \ref{system} shows code for a program which sends a command string to the shell. The command string both invokes the command \textit{whoami} and redirects the output to a file called \textit{me.dat} line:9. This file is then read and the active user name is then displayed on line:25. The INTERACTION part shows that the \textit{system(...)} function was successful and that the user is \textit{proteus}.

\subsection{sprintf(...)}

\index{sprintf(...)}

\textit{sprintf(...)} is a variation of the \textit{printf(...)} function. Instead of directing the output to a stream, it directs the output to a string. This can be extremely useful for a number of activities including building dynamic strings like file paths. Care has to be given that there is enough space for the created string.

\begin{lstlisting}[language=C,showstringspaces=false, caption={File: sprintf.c},captionpos=b,label=sprintf]

 1 #include <stdio.h>
 2 
 3 void new_file_path(char *filepath, char *filename)
 4 {
 5 sprintf(filepath,"project/special/%s",filename);
 6 }
 7 
 8 void clear(void)
 9 {    
10   while (getchar() != '\n' ); // clear stdin
11 }
12 
13 int main(void)
14 {
15 char project[40];
16 char filepath[256];
17 
18 printf("enter project  name?\n");
19 scanf("%s", project); // scan for user input
20 clear(); // clear buffer
21 new_file_path(filepath,project);
22 printf("filepath: %s\n",filepath);
23 
24 return 0;
25 }

INTERACTION

$ cc sprintf.c
$ ./a.out
enter project  name?
hello.c
filepath: project/special/hello.c
$

\end{lstlisting}

Listing \ref{sprintf} shows a program that takes user input, as shown on line:19. The input string \textit{project} is used to construct a file path string, shown on line:5. The program then outputs the newly created file path on line:22. In the INTERACTION part we can see that the user typed in \textit{hello.c}. The program then appended \textit{project/special/} onto the file name to give a complete file path of \textit{project/special/hello.c}.



